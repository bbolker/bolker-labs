\documentclass[]{article}
\usepackage{lmodern}
\usepackage{amssymb,amsmath}
\usepackage{ifxetex,ifluatex}
\usepackage{fixltx2e} % provides \textsubscript
\ifnum 0\ifxetex 1\fi\ifluatex 1\fi=0 % if pdftex
  \usepackage[T1]{fontenc}
  \usepackage[utf8]{inputenc}
\else % if luatex or xelatex
  \ifxetex
    \usepackage{mathspec}
  \else
    \usepackage{fontspec}
  \fi
  \defaultfontfeatures{Ligatures=TeX,Scale=MatchLowercase}
\fi
% use upquote if available, for straight quotes in verbatim environments
\IfFileExists{upquote.sty}{\usepackage{upquote}}{}
% use microtype if available
\IfFileExists{microtype.sty}{%
\usepackage{microtype}
\UseMicrotypeSet[protrusion]{basicmath} % disable protrusion for tt fonts
}{}
\usepackage[margin=1in]{geometry}
\usepackage{hyperref}
\hypersetup{unicode=true,
            pdftitle={Additional exercise Ch. 6},
            pdfauthor={Bob Douma},
            pdfborder={0 0 0},
            breaklinks=true}
\urlstyle{same}  % don't use monospace font for urls
\usepackage{color}
\usepackage{fancyvrb}
\newcommand{\VerbBar}{|}
\newcommand{\VERB}{\Verb[commandchars=\\\{\}]}
\DefineVerbatimEnvironment{Highlighting}{Verbatim}{commandchars=\\\{\}}
% Add ',fontsize=\small' for more characters per line
\usepackage{framed}
\definecolor{shadecolor}{RGB}{248,248,248}
\newenvironment{Shaded}{\begin{snugshade}}{\end{snugshade}}
\newcommand{\KeywordTok}[1]{\textcolor[rgb]{0.13,0.29,0.53}{\textbf{{#1}}}}
\newcommand{\DataTypeTok}[1]{\textcolor[rgb]{0.13,0.29,0.53}{{#1}}}
\newcommand{\DecValTok}[1]{\textcolor[rgb]{0.00,0.00,0.81}{{#1}}}
\newcommand{\BaseNTok}[1]{\textcolor[rgb]{0.00,0.00,0.81}{{#1}}}
\newcommand{\FloatTok}[1]{\textcolor[rgb]{0.00,0.00,0.81}{{#1}}}
\newcommand{\ConstantTok}[1]{\textcolor[rgb]{0.00,0.00,0.00}{{#1}}}
\newcommand{\CharTok}[1]{\textcolor[rgb]{0.31,0.60,0.02}{{#1}}}
\newcommand{\SpecialCharTok}[1]{\textcolor[rgb]{0.00,0.00,0.00}{{#1}}}
\newcommand{\StringTok}[1]{\textcolor[rgb]{0.31,0.60,0.02}{{#1}}}
\newcommand{\VerbatimStringTok}[1]{\textcolor[rgb]{0.31,0.60,0.02}{{#1}}}
\newcommand{\SpecialStringTok}[1]{\textcolor[rgb]{0.31,0.60,0.02}{{#1}}}
\newcommand{\ImportTok}[1]{{#1}}
\newcommand{\CommentTok}[1]{\textcolor[rgb]{0.56,0.35,0.01}{\textit{{#1}}}}
\newcommand{\DocumentationTok}[1]{\textcolor[rgb]{0.56,0.35,0.01}{\textbf{\textit{{#1}}}}}
\newcommand{\AnnotationTok}[1]{\textcolor[rgb]{0.56,0.35,0.01}{\textbf{\textit{{#1}}}}}
\newcommand{\CommentVarTok}[1]{\textcolor[rgb]{0.56,0.35,0.01}{\textbf{\textit{{#1}}}}}
\newcommand{\OtherTok}[1]{\textcolor[rgb]{0.56,0.35,0.01}{{#1}}}
\newcommand{\FunctionTok}[1]{\textcolor[rgb]{0.00,0.00,0.00}{{#1}}}
\newcommand{\VariableTok}[1]{\textcolor[rgb]{0.00,0.00,0.00}{{#1}}}
\newcommand{\ControlFlowTok}[1]{\textcolor[rgb]{0.13,0.29,0.53}{\textbf{{#1}}}}
\newcommand{\OperatorTok}[1]{\textcolor[rgb]{0.81,0.36,0.00}{\textbf{{#1}}}}
\newcommand{\BuiltInTok}[1]{{#1}}
\newcommand{\ExtensionTok}[1]{{#1}}
\newcommand{\PreprocessorTok}[1]{\textcolor[rgb]{0.56,0.35,0.01}{\textit{{#1}}}}
\newcommand{\AttributeTok}[1]{\textcolor[rgb]{0.77,0.63,0.00}{{#1}}}
\newcommand{\RegionMarkerTok}[1]{{#1}}
\newcommand{\InformationTok}[1]{\textcolor[rgb]{0.56,0.35,0.01}{\textbf{\textit{{#1}}}}}
\newcommand{\WarningTok}[1]{\textcolor[rgb]{0.56,0.35,0.01}{\textbf{\textit{{#1}}}}}
\newcommand{\AlertTok}[1]{\textcolor[rgb]{0.94,0.16,0.16}{{#1}}}
\newcommand{\ErrorTok}[1]{\textcolor[rgb]{0.64,0.00,0.00}{\textbf{{#1}}}}
\newcommand{\NormalTok}[1]{{#1}}
\usepackage{graphicx,grffile}
\makeatletter
\def\maxwidth{\ifdim\Gin@nat@width>\linewidth\linewidth\else\Gin@nat@width\fi}
\def\maxheight{\ifdim\Gin@nat@height>\textheight\textheight\else\Gin@nat@height\fi}
\makeatother
% Scale images if necessary, so that they will not overflow the page
% margins by default, and it is still possible to overwrite the defaults
% using explicit options in \includegraphics[width, height, ...]{}
\setkeys{Gin}{width=\maxwidth,height=\maxheight,keepaspectratio}
\IfFileExists{parskip.sty}{%
\usepackage{parskip}
}{% else
\setlength{\parindent}{0pt}
\setlength{\parskip}{6pt plus 2pt minus 1pt}
}
\setlength{\emergencystretch}{3em}  % prevent overfull lines
\providecommand{\tightlist}{%
  \setlength{\itemsep}{0pt}\setlength{\parskip}{0pt}}
\setcounter{secnumdepth}{0}
% Redefines (sub)paragraphs to behave more like sections
\ifx\paragraph\undefined\else
\let\oldparagraph\paragraph
\renewcommand{\paragraph}[1]{\oldparagraph{#1}\mbox{}}
\fi
\ifx\subparagraph\undefined\else
\let\oldsubparagraph\subparagraph
\renewcommand{\subparagraph}[1]{\oldsubparagraph{#1}\mbox{}}
\fi

%%% Use protect on footnotes to avoid problems with footnotes in titles
\let\rmarkdownfootnote\footnote%
\def\footnote{\protect\rmarkdownfootnote}

%%% Change title format to be more compact
\usepackage{titling}

% Create subtitle command for use in maketitle
\newcommand{\subtitle}[1]{
  \posttitle{
    \begin{center}\large#1\end{center}
    }
}

\setlength{\droptitle}{-2em}
  \title{Additional exercise Ch. 6}
  \pretitle{\vspace{\droptitle}\centering\huge}
  \posttitle{\par}
  \author{Bob Douma}
  \preauthor{\centering\large\emph}
  \postauthor{\par}
  \predate{\centering\large\emph}
  \postdate{\par}
  \date{22 November 2017}


\begin{document}
\maketitle

\subsection{Fitting models to data}\label{fitting-models-to-data}

In this exercise you will learn how to fit models to data through means
of maximum likelihood and compare the likelihood of different models
(hypotheses). Fitting a model to data through likelihood requires that
you take four steps:

\begin{enumerate}
\def\labelenumi{\arabic{enumi}.}
\tightlist
\item
  Specify how the dependent variable depends on the independent
  variable, i.e.~specify a function how the mean of y depends on the
  value of x.
\item
  Specify a probability distribution to describe the deviations of the
  observations from the mean
\item
  Choose the parameters of the deterministic model and the probability
  model such that the negative log likelihood is lowest.
\item
  Compare the likelihood of alternative models (change the deterministic
  function or the stochastic function) and compare with AIC(c) or BIC
  which model is most parsimonious.
\end{enumerate}

To fit a model through means of maximum likelihood you need to specify a
function that calculate the negative log likelihood (NLL) based on the
data and the parameter values. For example to calculate the NLL of a
linear model and a normal distribution the following function works:

\begin{Shaded}
\begin{Highlighting}[]
\NormalTok{nll =}\StringTok{ }\NormalTok{function(par,y,x)\{}
  \NormalTok{a =}\StringTok{ }\NormalTok{par[}\DecValTok{1}\NormalTok{]}
  \NormalTok{b =}\StringTok{ }\NormalTok{par[}\DecValTok{2}\NormalTok{]}
  \NormalTok{sd =}\StringTok{ }\NormalTok{par[}\DecValTok{3}\NormalTok{]}
  \CommentTok{# this calculates the mean y for a given value of x: the deterministic function}
  \NormalTok{mu =}\StringTok{ }\NormalTok{a+b*x }
  \CommentTok{# this calculates the likelihood of the function given the probability }
  \CommentTok{# distribution, the data and mu and sd}
  \NormalTok{nll =}\StringTok{ }\NormalTok{-}\KeywordTok{sum}\NormalTok{(}\KeywordTok{log}\NormalTok{(}\KeywordTok{dnorm}\NormalTok{(y,}\DataTypeTok{mean=}\NormalTok{mu,}\DataTypeTok{sd=}\NormalTok{sd))) }
  \KeywordTok{return}\NormalTok{(nll)}
\NormalTok{\}}
\end{Highlighting}
\end{Shaded}

Next we specify a function to find the maximum likelihood estimate

\begin{Shaded}
\begin{Highlighting}[]
\NormalTok{par=}\KeywordTok{c}\NormalTok{(}\DataTypeTok{a=}\DecValTok{1}\NormalTok{,}\DataTypeTok{b=}\DecValTok{1}\NormalTok{,}\DataTypeTok{c=}\DecValTok{1}\NormalTok{) }\CommentTok{# initial parameters}
\NormalTok{opt1 =}\StringTok{ }\KeywordTok{optim}\NormalTok{(}\DataTypeTok{par=}\NormalTok{par,nll,}\DataTypeTok{x=}\NormalTok{x,}\DataTypeTok{y=}\NormalTok{y) }\CommentTok{# y represents the data, x the independent variable}
\end{Highlighting}
\end{Shaded}

It can also be done through \texttt{mle2}

\begin{Shaded}
\begin{Highlighting}[]
\NormalTok{nll.mle =}\StringTok{ }\NormalTok{function(a,b,sd)\{}
  \CommentTok{# this calculates the mean y for a given value of x: the deterministic function}
  \NormalTok{mu =}\StringTok{ }\NormalTok{a+b*x }
  \CommentTok{# this calculates the likelihood of the function given the probability }
  \CommentTok{# distribution, the data and mu and sd}
  \NormalTok{nll =}\StringTok{ }\NormalTok{-}\KeywordTok{sum}\NormalTok{(}\KeywordTok{log}\NormalTok{(}\KeywordTok{dnorm}\NormalTok{(y,}\DataTypeTok{mean=}\NormalTok{mu,}\DataTypeTok{sd=}\NormalTok{sd))) }
  \KeywordTok{return}\NormalTok{(nll)}
\NormalTok{\}}
\end{Highlighting}
\end{Shaded}

\begin{Shaded}
\begin{Highlighting}[]
\CommentTok{# the data should be supplied through data and the parameters through list().}
\NormalTok{mle2}\FloatTok{.1} \NormalTok{=}\StringTok{ }\KeywordTok{mle2}\NormalTok{(nll.mle,}\DataTypeTok{start=}\KeywordTok{list}\NormalTok{(}\DataTypeTok{a=}\DecValTok{1}\NormalTok{,}\DataTypeTok{b=}\DecValTok{1}\NormalTok{,}\DataTypeTok{sd=}\DecValTok{1}\NormalTok{),}\DataTypeTok{data=}\KeywordTok{data.frame}\NormalTok{(x,y)) }
\KeywordTok{summary}\NormalTok{(mle2}\FloatTok{.1}\NormalTok{)}
\end{Highlighting}
\end{Shaded}

The following steps will lead you through the model fitting procedure.

\begin{enumerate}
\def\labelenumi{\arabic{enumi}.}
\tightlist
\item
  Take the first dataset and tweak the above functions such that it
  matches with the deterministic and stochastic model that you have
  chosen. In case you got stuck in the previous exercises where you had
  to choose a deterministic function and a stochastic function see next
  page for suggestions. \emph{hint}: In a previous exercise you have
  eyeballed the parameter values of the functions, you can use these as
  starting values. \emph{hint}: In case you get convergence problems,
  adapt your starting values, or choose a different optimizer. For
  example Nelder-Mead is a robust one, e.g.
  \texttt{method\ =\ "Nelder-Mead"}.
\item
  Change the determinstic function for a possible alternative
  determinstic function
\item
  Compare the likelihoods of the data given both models
\item
  Apply model selection criteria and conclude which model fits that data
  best.
\item
  Does the model makes sense from a biological perspective?
\end{enumerate}

Optional and time permitting:\\
6. Repeat the above procedure for the other 5 datasets

\newpage

\subsection{Hints for choosing deterministic functions and stochastic
functions}\label{hints-for-choosing-deterministic-functions-and-stochastic-functions}

\begin{enumerate}
\def\labelenumi{\arabic{enumi}.}
\item
  Deterministic functions

  \textbf{dataset 1} light response curve. There are a number of options
  of functions to choose from, depending on the level of sophistication:
  \(\frac{ax}{(b+x)}\), \(a(1-e^{(-bx)})\),
  \(\frac{1}{2\theta}(\alpha I+p_{max}-\sqrt(\alpha I+p_{max})^2-4\theta I p_{max})\)
  see page 98.

  \textbf{dataset 2} The dataset describes a functional responses.
  Bolker mentions four of those \(min(ax,s)\) \(\frac{ax}{(b+x)}\),
  \(\frac{ax^2}{(b^2+x^2)}\),\(\frac{ax^2}{(b+cx+x^2)}\)

  \textbf{dataset 3} Allometric relationships generally have the form
  \(ax^b\)

  \textbf{dataset 4} This could be logistic growth
  \(n(t)=\frac{K}{1+(\frac{K}{n_0})e^{-rt}}\) or the gompertz function
  \(f(x)=e^{-ae^{-bx}}\)

  \textbf{dataset 5} What about a negative exponential? \(ae{-bx}\) or a
  power function \(ax^b\)

  \textbf{dataset 6} Species reponse curves are curves that describe the
  probability of presence as a function of some factor. A good candidate
  good be a unimodel response curve. You could take the equation of the
  normal distribution without the scaling constant: e.g.
  \(a e^{\frac{-(x-\mu)^2}{2\sigma^2}}\)
\item
  Stochastic functions/Probability distributions

  \textbf{dataset 1} y represents real numbers and both positive and
  negative numbers occur. This implies that we should choose a
  continuous probability distribution. In addition, the numbers seems
  unbound. Within the family of continuous probability distributions,
  the normal seems a good candidate distribution because this one runs
  from -\(\inf\) to +\(\inf\). In contrast the Gamma and the Lognormal
  only can take positive numbers, so these distributions cannot handle
  the negative numbers. In addition, the beta distribution is not a good
  candidate because it runs from 0-1.

  \textbf{dataset 2} y represents real numbers and only positive numbers
  occur. The data represents a functional response (intake rate of the
  predator), and it is likely that you can only measure positive numbers
  (number of prey items per unit of time). This implies that we should
  choose a continuous probability distribution. Within the family of
  continuous probability distributions, the Gamma and the Lognormal
  could be taken as candidate distributions because they can only take
  positive numbers (beware that the Gamma cannot take 0). However, you
  could try to use a normal as well.

  \textbf{dataset 3} y seems represents counts (this is the cone dataset
  that is introduced in ch.~6.). Given that it contains counts we can
  pick a distribution from the family of discrete distributions. The
  Poisson and the Negative Binomial could be good candidates to describe
  this type of data.

  \textbf{dataset 4} y represents population size over time. From
  looking at the data, they seems to represent counts. Given that it
  contains counts we can pick a distribution from the family of discrete
  distributions. The Poisson and the Negative Binomial could be good
  candidates to describe this type of data.

  \textbf{dataset 5} No information is given on y. The data clearly
  seems to represent counts. Thus the same reasoning applies here as to
  the two previous datasets.

  \textbf{dataset 6} The data (y) represents species occurences
  (presence/absence). The binomial model would be a good model to
  predict the probability of presence.
\end{enumerate}


\end{document}
